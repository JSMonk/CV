% Copyright 2013 Christophe-Marie Duquesne <chmd@chmd.fr>
% Copyright 2014 Mark Szepieniec <http://github.com/mszep>
% 
% ConText style for making a resume with pandoc. Inspired by moderncv.
% 
% This CSS document is delivered to you under the CC BY-SA 3.0 License.
% https://creativecommons.org/licenses/by-sa/3.0/deed.en_US

\startmode[*mkii]
  \enableregime[utf-8]  
  \setupcolors[state=start]
\stopmode

\setupcolor[hex]
\definecolor[titlegrey][h=757575]
\definecolor[sectioncolor][h=397249]
\definecolor[rulecolor][h=9cb770]

% Enable hyperlinks
\setupinteraction[state=start, color=sectioncolor]

\setuppapersize [A4][A4]
\setuplayout    [width=middle, height=middle,
                 backspace=20mm, cutspace=0mm,
                 topspace=10mm, bottomspace=20mm,
                 header=0mm, footer=0mm]

%\setuppagenumbering[location={footer,center}]

\setupbodyfont[11pt, helvetica]

\setupwhitespace[medium]

\setupblackrules[width=31mm, color=rulecolor]

\setuphead[chapter]      [style=\tfd]
\setuphead[section]      [style=\tfd\bf, color=titlegrey, align=middle]
\setuphead[subsection]   [style=\tfb\bf, color=sectioncolor, align=right,
                          before={\leavevmode\blackrule\hspace}]
\setuphead[subsubsection][style=\bf]

\setuphead[chapter, section, subsection, subsubsection][number=no]

%\setupdescriptions[width=10mm]

\definedescription
  [description]
  [headstyle=bold, style=normal,
   location=hanging, width=18mm, distance=14mm, margin=0cm]

\setupitemize[autointro, packed]    % prevent orphan list intro
\setupitemize[indentnext=no]

\setupfloat[figure][default={here,nonumber}]
\setupfloat[table][default={here,nonumber}]

\setuptables[textwidth=max, HL=none]
\setupxtable[frame=off,option={stretch,width}]

\setupthinrules[width=15em] % width of horizontal rules

\setupdelimitedtext
  [blockquote]
  [before={\setupalign[middle]},
   indentnext=no,
  ]


\starttext

\section[title={Artem Kobzar 111},reference={artem-kobzar-111}]

\startplacetable[location=none]
\startxtable
\startxtablebody[body]
\startxrow
\startxcell[align=right] Kharkiv, Ukraine (ready to relocate) \stopxcell
\stopxrow
\startxrow
\startxcell[align=right] Email:
\useURL[url1][mailto:a.kobzar.nlt@gmail.com][][a.kobzar.nlt@gmail.com]\from[url1] \stopxcell
\stopxrow
\startxrow
\startxcell[align=right] GitHub:
\useURL[url2][https://github.com/JSMonk][][JSMonk]\from[url2] \stopxcell
\stopxrow
\stopxtablebody
\startxtablefoot[foot]
\startxrow
\startxcell[align=right] Twitter:
\useURL[url3][https://twitter.com/rage_monk][][\type{@rage_monk}]\from[url3] \stopxcell
\stopxrow
\stopxtablefoot
\stopxtable
\stopplacetable

\subsection[title={Education},reference={education}]

\startdescription{2019-2021 (expected)}
  {\bf MSc, Software Engineering};
  \useURL[url4][http://www.kpi.kharkov.ua/eng/][][National Technical
  University \quotation{Kharkiv Polytechnic Institute}]\from[url4]
\stopdescription

\startdescription{2015-2019}
  {\bf BSc, Computer Engineering};
  \useURL[url5][http://www.kpi.kharkov.ua/eng/][][National Technical
  University \quotation{Kharkiv Polytechnic Institute}]\from[url5]
\stopdescription

\subsection[title={Experience},reference={experience}]

{\bf The Most Recent Work Experience:}

I worked (from September 2019 to May 2020) at
\useURL[url6][https://www.wrike.com/][][Wrike Company]\from[url6] in a
new Research and Development Team.

I and my teammates (other 2 people) built development analytic process
and developed infrastructure for the process from scratch.

My responsibilities areas in the work were:

\startitemize
\item
  Analysis platform.
\item
  Knowledgebase (an API which store project attributes).
\item
  DX Improvement Tools (editor extensions, GitLab webhooks, etc)
\stopitemize

We already got results: defined separated responsibility area for our
teams, which improved the development process (fast search and
communication with code owner) and more productive replacing of
deprecated technologies from the product.

{\bf Other Recent Job I Had}

\startitemize
\item
  \useURL[url7][https://hy.dev/][][HellYeah]\from[url7] Research
  Department. I worked on cryptography libraries related to digital
  signatures and private cryptocurrencies clients with high-level
  safety. As a result of research, I developed zero-dependency
  cryptocurrencies clients, which currently used in private
  projects(NDA).
\item
  \useURL[url8][https://www.upwork.com/o/companies/~019c0fc838498df613/][][WookieeLabs]\from[url8].
  \useURL[url9][https://www.upwork.com/o/profiles/users/~01743ab09e751efe1c/][][My
  Upwork profile at the company]\from[url9]. I worked at different
  outsource projects as a software engineer:
  \useURL[url10][https://www.primero.org/][][Unicef
  Primero]\from[url10],
  \useURL[url11][https://www.nextlead.io/][][NextLead]\from[url11],
  \useURL[url12][https://crossoverhealth.com/][][Crossover
  Health]\from[url12].
\stopitemize

\subsection[title={Technical Skills},reference={technical-skills}]

{\bf \quotation{Primary} Technologies}

I commonly work with \type{JavaScript} ecosystem.

Especially, I worked a lot with
\useURL[url13][https://nodejs.org/][][Node.js]\from[url13] and popular
backend frameworks like
(\useURL[url14][https://nestjs.com/][][NestJS]\from[url14],
\useURL[url15][https://expressjs.com/ru/][][Express]\from[url15],
\useURL[url16][https://koajs.com/][][Koa]\from[url16],
\useURL[url17][https://sailsjs.com/][][Sails]\from[url17]) at the
backend side.

At the backend, I used different DB solutions:
\useURL[url18][https://www.mysql.com/][][MySQL]\from[url18],
\useURL[url19][https://www.sqlite.org/][][SQLite]\from[url19],
\useURL[url20][https://www.postgresql.org/][][PostgreSQL]\from[url20],
\useURL[url21][https://www.mongodb.com/][][MongoDB]\from[url21],
\useURL[url22][https://redis.io/][][Redis]\from[url22],
\useURL[url23][https://aws.amazon.com/dynamodb][][DynamoDB]\from[url23].

Additionally, I worked a lot with popular front-end frameworks like
\useURL[url24][https://reactjs.org/][][React]\from[url24]
(\useURL[url25][https://reactnative.dev/][][React Native]\from[url25]
too) and \useURL[url26][https://vuejs.org/][][Vue.js]\from[url26].

And of course, I worked with popular static type solutions for
JavaScript:
\useURL[url27][https://www.typescriptlang.org/][][TypeScript]\from[url27]
and \useURL[url28][https://flow.org/][][Flow.js]\from[url28].

{\bf \quotation{Secondary} Technologies}

I worked with \type{C/C++} and
\useURL[url29][https://www.rust-lang.org/][][Rust]\from[url29]
ecosystem, especially when I was working with cryptography and
cryptocurrencies.

Also, I worked with \useURL[url30][https://dart.dev/][][Dart
Language]\from[url30] and his ecosystem, when I was working in
\useURL[url31][https://www.wrike.com/][][Wrike Company]\from[url31].

At the start of my programming career, I worked with \type{C#} and
\type{.NET}, especially, with \type{.NET MVC} and \type{.NET Web API}.

And when I was a freelancer with pure
\useURL[url32][https://www.php.net/][][PHP]\from[url32] and
\useURL[url33][https://www.ruby-lang.org/ru/][][Ruby]\from[url33]
(\useURL[url34][https://rubyonrails.org/][][RoR]\from[url34] and
\useURL[url35][http://sinatrarb.com/][][Sinatra]\from[url35]).

{\bf \quotation{Basic Knowledge} Technologies}

In my spare time, I really like to work with:
\useURL[url36][https://www.haskell.org/][][Haskell]\from[url36],
\useURL[url37][https://www.idris-lang.org/][][Idris]\from[url37],
\useURL[url38][https://reasonml.github.io/][][ReasonML]\from[url38]/\useURL[url39][https://ocaml.org/][][OCaml]\from[url39],
\useURL[url40][https://elixir-lang.org/][][Elixir]\from[url40],
\useURL[url41][https://clojure.org/][][Clojure]\from[url41] and
\useURL[url42][https://golang.org/][][Golang]\from[url42]. Additionally,
I basically know
\useURL[url43][https://www.java.com][][Java]\from[url43],
\useURL[url44][https://www.apple.com/swift/][][Swift]\from[url44],
\useURL[url45][https://www.python.org/][][Python]\from[url45].

\subsection[title={Open Source},reference={open-source}]

{\bf My Own}

\startitemize
\item
  \useURL[url46][https://github.com/JSMonk/hegel][][Hegel]\from[url46] -
  a static type checker for JavaScript with high-level type inference
  (Hindley-Milner algorithm) and a strong type system.
\item
  \useURL[url47][https://github.com/JSMonk/sweet-monads][][sweet-monads]\from[url47]
  - zero-dependency monads for JavaScript.
\stopitemize

{\bf In which I make significant changes}

\startitemize
\item
  \useURL[url48][https://github.com/paulmillr/chokidar][][chokidar]\from[url48]
  - the most popular Node.js file watcher. The 3.0.0 version was
  developed on my own.
\item
  \useURL[url49][https://github.com/paulmillr/readdirp][][readdirp]\from[url49]
  - recursive \type{readdir} with additional goodies (15,603,219
  downloads per week). The 3.0.0 version was developed on my own.
\item
  \useURL[url50][https://github.com/paulmillr/noble-ripemd160][][noble-ripemd160]\from[url50]
  - a cryptography hash function.
\item
  \useURL[url51][https://github.com/paulmillr/noble-ed25519][][noble-ed25519]\from[url51]
  - a digital signature library based at ed15519 schema.
\item
  \useURL[url52][https://github.com/paulmillr/noble-secp256k1][][noble-secp256k1]\from[url52]
  - a digital signature library based at secp256k1 schema.
\item
  \useURL[url53][https://github.com/paulmillr/noble-bls12-381][][noble-bls12-381]\from[url53]
  - a digital signature library based at noble12-381 schema.
\stopitemize

\startblockquote
Initially, noble projects were developed by me and after were handed
over to \useURL[url54][https://github.com/paulmillr][][Paul
Miller]\from[url54]:
\stopblockquote

\subsection[title={Other},reference={other}]

\startitemize
\item
  I teach students at
  \useURL[url55][http://javascript.ninja/][][JavaScript.Ninja
  Project]\from[url55]
\item
  I and my friend make a Russian-language podcast
  \useURL[url56][https://underjs.ru/][][UnderJS]\from[url56] about
  JavaScript and Compilers.
\item
  I also one of the organizers one of the largest conference in
  Russian-language countries
  \useURL[url57][https://holyjs.ru/][][HolyJS]\from[url57]
\item
  And also, I'm a public speaker. For now, I speak only for the Russian
  language audience. My recent talks:
  \useURL[url58][https://www.youtube.com/watch?v=JKvmwOuqVWI][][\quotation{Dissection
  of Dart VM}]\from[url58],
  \useURL[url59][https://www.youtube.com/watch?v=S0cCjbWuvzk&t=39s][][\quotation{Either
  monad: theory and practice}]\from[url59],
  \useURL[url60][https://www.youtube.com/watch?v=GIHrPm_YAIc&t=1715s][][\quotation{Why
  and How I write one more static type checker}]\from[url60].
\item
  English level: B1/B2
\stopitemize

\stoptext
